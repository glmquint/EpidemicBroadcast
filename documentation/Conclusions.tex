\section{Conclusions}
With this analysis we've shown the varying influences that each parameter has on system performance. 

We can state that in an ideal scenario, having a very large radius that allows every node to connect with almost every other part of the graph is  enough to respect every performance index that we considered (section \ref{sec:perf-index}). The system is therefore fast and reliably delivers the message across the entire floorplan (section \ref{time-evolution}). A high probability of transmission is therefore encouraged for a faster completion.

Unfortunately, real systems often cannot satisfy this unrealistic assumption: for physical reasons, often related to system power and distances, nodes are limited by a very narrow view of the system, that can often lead to a higher chance of dealing with a lightly connected or even unconnected graph. In this cases, system engineers should tune down the probability threshold below $50\%$, which drastically reduces collisions and increases the possibility to reach all nodes (Fig. \ref{fig:CovPercBoxplotNodes}). Assuming we require to complete with a total coverage percentage, runs with lower probability and a high enough radius to avoid unconnected graph cases perform the best (Fig. \ref{fig:200-90CI}). In case we prefer a faster system response in favor of a lower final coverage, it is possible to employ a higher probability threshold, with a global trend as shown in section \ref{50-and-700-cases}.

Regarding the assumptions made for this analysis (section \ref{params-and-assumtions}) it would be possible to extend the range of other floorplan configurations, especially the condition of non-uniformly distributed nodes. This would allow the modelling of sub-networks inside the whole graph, with the added effect of certain nodes having the role of a 'bridge' between groups and probably a higher chance of resulting in a bottleneck.

This analysis explored the variation of multiple parameters and we often choose to consider very large extremes. For specific cases, it would be much beneficial to focus the research on a narrower range of values, increasing the resolution for each parameter of interest.